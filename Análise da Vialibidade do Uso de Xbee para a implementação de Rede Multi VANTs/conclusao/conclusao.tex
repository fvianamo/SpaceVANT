\mychapter{Conclus\~{a}o}
\label{Cap:conclusao}

Ao realizar uma análise desse trabalho, pode-se afirmar que o mesmo sucedeu-se de maneira satisfatória, sendo possível afirmar que os principais objetivos foram cumpridos. 

Durante a execução do trabalho, foram realizadas pesquisas acerca de sistemas industriais, controladores lógicos programáveis, sistemas de supervisão e instrumentação, além da realização de experiências práticas com equipamentos eletrônicos e industriais. Sendo assim, foi possível identificar características, vantagens e desvantagens da utilização de diferentes equipamentos e tecnologias para desenvolvimento de uma aplicação industrial tal qual a descrita nesse projeto.

Foi estabelecida a comunicação entre o CLP TPW03 da fabricante WEG, o controlador de processos N2000 da NOVUS e um sistema de supervisão desenvolvido no Elipse SCADA de maneira eficiente, testada com a leitura de sensores de temperatura.

Todo o processo desenvolvido durante o trabalho foi documentado e demonstrado de maneira objetiva, de modo que é possível que outras pessoas podem seguir passo a passo o que foi feito e conseguir criar um sistem supervisório simples e estabelecer uma comunicação entre ele e diferentes tipos de CLPs, apenas tendo em mente às diferenças entre um equipamento e outro, tais quais: a configuração dos parâmetros de comunicação, a observação dos endereços Modbus (que variam para cada equipamento) e as funções utilizadas para realizar a leitura, escrita dentre outras operações sobre uma \textit{tag} associada à um equipamento.

\section{Dificuldades Encontradas}

Muito do que foi desenvolvido durante esse projeto baseou-se em tecnologias que variavam de acordo com cada fabricante, devido à isso, uma grande quantidade de tempo foi investida somente entendendo manuais de instalação e programação de cada equipamento e software. Além disso, a montagem do sistema também demandava tempo, visto que a quantidade de ligações entre elementos era considerável e algumas vezes necessitavam ser desfeitas.

\section{Trabalhos Futuros}

Através do que foi descrito nesse trabalho, em continuidade à ele, é possível desenvolver um sistema supervisório mais completo que capte outros dados da planta industrial presente nas instalações do LAMP. Assim, pode-se realizar o monitoramento de todos os sensores utilizando as mesmas técnicas, somente ampliando o sistema para a leitura de um número superior de sensores. Além disso, aplicando equipamentos extras ao sistema, como módulos de expansões que contenham saídas analógicas, também torna-se possível a manipulação de válvulas através do sistema SCADA. 

Nesse trabalho não foram considerados testes de entradas e saídas digitais, porém a lógica para a leitura e escrita dessas portas é similar à leitura e escrita analógica no Elipse SCADA, diferindo apenas em relação às operações, sendo necessária a seleção da função adequada à operações com variáveis digitais no software da Elipse. Desse modo, pode-se também inserir elementos digitais ao sistema como botões e chaves, por exemplo.

\subsection{Outras Técnicas de Comunicação}

Atualmente, o protocolo Modbus possui também um modo de operação variante via Ethernet, o chamado Modbus TCP/IP que é baseado na utilização dos protocolos TCP/IP. Esse tipo de protocolo também pode ser considerado para a obtenção de dados de sistemas industriais em substituição às conexões via RS-485 e RS-232. Uma consideração futura seria estabelecer um estudo da necessidade e viabilidade de uma conexão desse tipo nos sistemas supervisórios a serem desenvolvidos. Estudos como esse podem partir de softwares similares aos descritos aqui, realizando a alteração do modo de execução Serial para Ethernet no \textit{driver} Modbus.

\subsection{CLPs como mestres de uma rede Modbus}

Nesse trabalho, o software de supervisão mestre é desenvolvido na plataforma Elipse SCADA, porém outros sistemas, pode se desejar a atuação do software supervisório como escravo de uma rede Modbus. Para uma rede em que se tenham diveros CLPs, pode ser interessante a comunicação entre um CLP mestre e outros controladores escravos e posteriormente a ligação do CLP (mestre em relação aos outros CLPs) com o sistema supervisório. Nessa situação, o supervisório estaria agindo inicialmente como escravo da comunicação e posteriormente como mestre. Para que o Elipse SCADA atue como escravo de uma comunicação Modbus, deve haver a substituição do \textit{driver Modcon Modbus} pelo driver \textit{Modbus Slave}\cite{elipse2016modbus} também disponibilizado pela Elipse.






