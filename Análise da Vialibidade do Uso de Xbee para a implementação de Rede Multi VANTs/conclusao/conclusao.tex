\mychapter{Conclus\~{a}o}
\label{Cap:conclusao}

Apesar do local de realização dos testes não ter sido o ideal e alguns dos dados coletados não corresponderem com o que era esperado, os testes descritos nesse trabalho são de relevância para o estudo que está sendo realizado pelo mestrando Maurício Rabelo e, consequentemente, para o projeto SpaceVANT.

A partir dos testes realizados na própria UFRN, mais especificamente no campo de futebol central, foi possível identificar diversos parâmetros que interferem na taxa de sucesso de envio dos pacotes na rede XBee, parâmetros esses como: tamanho do \emph{payload} do pacote sendo enviado e a configuração de \emph{baudrate} do módulo XBee.

Sendo assim, por mais que os valores especificados no manual do dispositivo não tenham sido confirmados nos testes, principalmente nos testes em solo, os resultados se mostraram promissores. No entanto, ainda é prematuro dizer que o módulo XBee PRO S3 900HP é adequado para implementação do sistema proposto pelo projeto SpaceVANT, mas acredita-se que essa premissa se confirmará com a realização de testes mais direcionados, de preferência realizados nas instalações do CLBI, analisando os efeitos de uma configuração mais refinada do módulo XBee e mensurando o impacto do tamanho da mensagem transmitida no perda de pacotes.     

 