\mychapter{Introdução}
\label{Cap:introducao}

A utilização de veículos não tripulados já é bastante evidente em aplicações tanto civis quanto militares. Podem-se encontrar veículos dessa categoria substituindo a presença humana em situações onde há risco a integridade física ou quando o acesso é simplesmente impossível. Dentre os veículos não tripulados, temos a categoria de veículos aéreos não tripulados (VANTs) que são usados largamente para realização de filmagens aérea a baixo custo. Com o investimento de algumas centenas de dólares, qualquer pessoa pode começar a produzir imagens aéreas utilizando VANTs comerciais. O mercado está repleto de modelos comerciais disponíveis para o público em geral, como por exemplo os quadrirrotores fabricados pela DJI, o recém anunciado Karma fabricado pela GoPro, entre outros.

Como citado anteriormente, as aplicações para veículos aéreos não tripulados não se restringe ao uso civil ou para gravação de imagens aéreas, esta plataforma já vem sendo utilizada também em aplicações militares. Ao aliar o poder da plataforma em questão com outras tecnologias, como por exemplo o processamento digital de imagens, problemas mais complexos podem ser resolvidos. 

Um problema que pode ser solucionado com a utilização de VANTs dotados de ferramentas para processamento digital de imagem seria a identificação de embarcações não autorizadas em área de impacto de foguetes, problema esse relevante ao Centro de Lançamento Barreira do Inferno (CLBI) localizada em Natal no Rio Grande do Norte. 

Em parceria com a Universidade Federal do Rio Grande do Norte (UFRN), através do projeto de pesquisa SPACEVANT coordenado pelo professor Dr. Pablo Javier Alsina, o CBLI vem desenvolvendo uma solução, incluindo software e hardware, para a realização da verificação da aérea de impacto de foguetes de forma autônoma utilizando VANTs. 

\section{Objetivos}

Como parte do desenvolvimento dessa solução, esse trabalho tem por objetivo validar as especificações técnicas do transmissor XBEE PRO S3 900HP adquirido para a implementação da rede de comunicação e a viabilidade da utilização desse tipo de equipamento no contexto de uma rede multi VANT.

A fim de realizar essa validação, foram realizados teste de força de sinal e taxa de transferência de pacotes em uma rede mesh/ad hoc, implementada por módulos XBee PRO S3 900HP, usando quadrirrotores Phantom 3 do modelo Standard fabricados pela DJI para variar a distância entre os pontos da rede e, posteriormente, verificar os efeitos do distanciamento nos parâmetros estudados.

\section{Estrutura do Trabalho}

Após este capitulo introdutório, é apresentada uma breve descrição do projeto SpaceVANT a fim de familiarizar o leitor com o contexto desse trabalho. Em seguida, no capítulo 3, são discutidos os requisitos aos quais uma rede multi VANTs deve atender. 

A estratégia de varredura de area desenvolvida pelo mestrando Maurício Rabello para o projeto é apresentada no capítulo 4. 

No capítulo 5, são apresentados o módulo Xbee adquirido pelo projeto para a implementação da rede, bem como o protocolo ZigBee. 

Em seguida, o procedimento experimental desenvolvido, bem como os equipamentos utilizados para a sua realização, são apresentados no capítulo 6.

Por fim temos a discussão dos resultados experimentais e a conclusão do trabalho nos capítulos 7 e 8, respectivamente.  