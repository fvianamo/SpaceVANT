% ********** P\'{a}gina de assinaturas
%

\begin{titlepage}
%
\begin{center}
%

%
\Large \textbf{Análise da Viabilidade do Uso de Xbee para a Implementação de um Rede Multi VANTs}


\vfill

\Large \textbf{Filipe Viana Monteiro}

\bigskip
\bigskip
\bigskip
\bigskip

\normalsize

Orientador: Prof. Dr. Pablo Javier Alsina

\vfill

\hfill
\parbox{0.5\linewidth}{
% Descomente as opes que se aplicam ao seu caso
\textbf{Monografia}
apresentada \`{a} Banca Examinadora do Trabalho de Conclus\~{a}o do Curso de
Engenharia de Computa\c{c}\~{a}o, em cumprimento \`{a}s exig\^{e}ncias
legais como requisito parcial \`{a} obten\c{c}\~{a}o do t\'{\i}tulo de Engenheiro de
Computa\c{c}\~{a}o.}

\vfill

\large

Natal/RN

Dezembro de 2016

\end{center}
\end{titlepage}

%
% ********** Dedicat\'{o}ria
%

% A dedicat\'{o}ria n\~{a}o \'{e} obrigat\'{o}ria. Se voc\^{e} tem algu\'{e}m ou algo que teve
% uma import\^{a}ncia fundamental ao longo do seu curso, pode dedicar a ele(a)
% este trabalho. Geralmente n\~{a}o se faz dedicat\'{o}ria a v\'{a}rias pessoas: para
% isso existe a se\c{c}\~{a}o de agradecimentos.
% Se n\~{a}o quiser dedicat\'{o}ria, basta excluir o texto entre
% \begin{titlepage} e \end{titlepage}

\begin{titlepage}

\vspace*{\fill}

\hfill
\begin{minipage}{0.5\linewidth}
\begin{flushright}
\large\it
Aos meus pais, Américo Monteiro Filho e Rosalba Viana Monteiro, que com todos os esforços, puderam me proporcionar a melhor educação possível.

\end{flushright}
\end{minipage}

\vspace*{\fill}

\end{titlepage}

%
% ********** Agradecimentos
%

% Os agradecimentos n\~{a}o s\~{a}o obrigat\'{o}rios. Se existem pessoas que lhe
% ajudaram ao longo do seu curso, pode incluir um agradecimento.
% Se n\~{a}o quiser agradecimentos, basta excluir o texto ap\'{o}s \chapter*{...}

\chapter*{Agradecimentos}
\thispagestyle{empty}

\begin{trivlist}  \itemsep 2ex

\item Primeiramente tenho que agradecer ao meus pais, Américo Monteiro Filho e Rosalba Viana Monteiro, que sempre me possibilitaram o melhor ensino que nos era acessível. Nunca esquecerei o esforço que fizeram, e ainda fazem, para educar a mim e a minha irmã Andreza. Espero um dia poder me tornar um pai tão bom quanto vocês.
  
\item Também devo agradecer aos amigos que sempre me apoiaram durante minha trajetória académica, seja ajudando a entender os conteúdos de sala de aula ou me ajudando a esquecer um pouco desses conteúdos e aliviar um pouco minha cabeça de toda a pressão que é a vida universitária.

\item Não posso esquecer do meu companheiro de testes, o mestrando Maurício Rabelo, que em muito me ajudou tanto na realização dos experimentos quanto na escrita desse trabalho, revisando meu texto e identificando oportunidades de melhoria.

\item Por fim tenho que agradecer, em especial, a excepcional turma de Engenharia da Computação formada em 2013, a qual faço parte. Acho que a nossa união foi um ponto que ajudou bastante na nossa formação. Nunca esquecerei das noites de estudos e projetos no DCA. E que venham muitas terças da pizza naquele mesmo lugar para comemorar nossas conquistas daqui pra frente.

\item Muito Obrigado a todos vocês.

\end{trivlist}

%\newpage

%\vspace*{\fill}
%\setlength{\epigraphrule}{1pt}
%\onehalfspacing
%\setlength{\epigraphwidth}{.95\textwidth}

%\begin{epigraphs}
%\qitem{
%    \textit{
%        \linebreak "Já dizia Tobias"
%        }
%    }
%{\textsc{Autor Desconhecido}}
%\end{epigraphs}



%\vspace*{\fill}
