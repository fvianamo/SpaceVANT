%
% ********** Pagina de Rosto
%

% titlepage gera paginas sem numera\c{c}ao
\begin{titlepage}

\begin{center}

\small

% O comando @{} no ambiente tabular x  para criar um novo delimitador
% entre colunas que no a barra vertical | que  normalmente utilizada.
% O delimitador desejado vai entre as chaves. No exemplo, no h nada,
% de modo que o delimitador  vazio. Este recurso est sendo usado para
% eliminar o espao que geralmente existe entre as colunas
\begin{tabularx}{\linewidth}{@{}l@{}C@{}r@{}}
% A figura foi colocada dentro de um parbox para que fique verticalmente
% centralizada em relao ao resto da linha
\parbox[c]{3cm}{\includegraphics[width=\linewidth]{LogoUFRN}} &
\begin{center}
\textsf{\textsc{Universidade Federal do Rio Grande do Norte\\
Centro de Tecnologia\\
Departamento de Engenharia de Computa\c{c}\~{a}o e Automa\c{c}\~{a}o\\
Curso de Engenharia de Computa\c{c}\~{a}o}}
\end{center}
\end{tabularx}


% O vfill  um espao vertical que assume a mxima dimenso possvel
% Os vfill's desta pgina foram utilizados para que o texto ocupe
% toda a folha
\vfill

\LARGE

\textbf{ANÁLISE DA VIABILIDADE DO USO DE XBEE PARA A IMPLEMENTAÇÃO DE UMA REDE MULTI VANTS}

\vfill

\Large

\textbf{Filipe Viana Monteiro}

\vfill
%
\normalsize

Orientador: Prof. Dr. Pablo Javier Alsina
% Se no houver co-orientador, comente a pr\'{o}xima linha
%\\[2ex] Co-orientador: Prof. Dr. Beltrano Catandura do Amaral

\vfill



%\textbf{Disserta\c{c}\~{a}o de Mestrado}
%\textbf{Tese de Doutorado}
%apresentada ao Programa de P\'{o}s-Gradu\c{c}\~{a}o em Engenharia El\'{e}trica da UFRN
%(\'{a}rea de concentra\c{c}\~{a}o: Automa\c{c}\~{a}o e Sistemas)
%(\'{a}rea de concentra\c{c}\~{a}o: Telecomunica\c{c}\~{o}es)
%como parte dos requisitos para obten\c{c}\~{a}o do t\'{\i}tulo de
%Mestre em Ci\^{e}ncias.}
%Doutor em Ci\^{e}ncias.}

\vfill

\large

Natal/RN

Dezembro de 2016

\end{center}

\end{titlepage}

