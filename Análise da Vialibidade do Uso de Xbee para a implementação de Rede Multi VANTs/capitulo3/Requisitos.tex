\mychapter{Características de uma Rede Multi VANTs}
\label{Cap:Requisitos}

As redes multi VANTs são uma evolução do sistema de comunicação formado por apenas um VANT e a base de controle, onde toda a troca de informações é realizada através de um \emph{link} entre as duas partes do sistema, situação em que há baixa confiabilidade no sistema de comunicação, pois caso esse \emph{link} seja interrompido por algum motivo, toda a comunicação seria impossibilitada. A rede multi VANTs é, então, proposta com intuito de aumentar o alcance e a confiabilidade do sistema através da adição de novos nós.

\cite{gupta2015survey} apresenta algumas das características de uma rede multi VANTs comparando-as à solução baseada em um único VANT, sendo elas:
\begin{itemize}
\item Escalabilidade
\item Capacidade de Sobrevivência
\item Velocidade da Missão
\item Complexidade do Controle
\item Economia de Energia
\end{itemize} 

\section{Escalabilidade}

Por definição uma rede multi VANTs é uma versão escalada da solução composta por um único VANT mais base de controle. Dessa forma, a adição de nós se torna uma operação comum a uma rede multi VANTs tornando-a uma solução escalável que pode variar o tamanho e, consequentemente, o alcance da rede dada as necessidades especificas da missão.

\section{Capacidade de Sobrevivência}

Como mencionado anteriormente, em uma solução formada por um único VANT mais base de controle, a rede é composta por apenas um \emph{link} e, caso esse seja interrompido, a comunicação entre esses dois nós será perdida. No entanto, dado que uma rede multi VANTs será formada por múltiplos nós e \emph{links}, em caso da perda de um nó a rede pode se reorganizar e garantir a sobrevivência da rede como um todo. 

Por exemplo, em uma rede composta por quatro nós em topologia \emph{mesh} como mostra a Figura \ref{fig:perdaNo}, no caso de falha do nó B, a comunicação entre os nós A e D poderá ser feita através do nó C, garantindo assim a sobrevivência da rede. 

\begin{figure} 
\center
\includegraphics[width=0.7\textwidth]{perdaNo.png}
\caption{Exemplo de topologia de uma rede multi VANTs composta por 4 nós.} 
\label{fig:perdaNo}
\end{figure} 

\section{Velocidade da Missão}

A velocidade de realização de uma determinada missão será proporcional ao tamanho da rede multi VANT, quanto maior a quantidade de VANTs em uma determinada rede a ser utilizada para executar uma missão de varredura de área, menor será o tempo da missão. Por exemplo, pode-se dividir a área a ser varrida em sub-áreas que serão varridas simultaneamente, alocando cada um dos VANTs disponíveis para a missão a uma sub-área diferente. Por outro lado, a adição de um nó a uma rede resultará no aumento do custo da missão. Dessa forma, é importante analisar os benefícios do aumento de uma rede multi VANTs do ponto de vista financeiro. 

\section{Complexidade do Controle}

A complexidade do controle de uma rede composta por apenas um VANT e uma base de controle é baixa quando comparado ao cenário de uma rede multi VANTs. Quanto maior a quantidade de nós em uma rede maior será o custo computacional para controlar os nós e determinar os caminhos para realização da comunicação entre eles.

Outro ponto que aumenta ainda mais essa complexidade é a frequente mudança de topologia da rede, seja causada pela movimentação rápida dos nós, que se dá nas três dimensões, ou pela perda de nós devido a problemas de funcionamento ou fim da carga da bateria que alimenta o VANT.

\section{Economia de Energia}

Como VANTs possuem fonte de alimentação limitada, é importante garantir que a rede funcione da forma mais eficiente possível quanto ao consumo de energia, de tal forma que os nós possam permanecer ativos na rede pelo máximo de tempo possível. Caso a rede de comunicação consuma muita energia dos VANTs, será necessário a substituição de nós durante a execução de uma missão o que, consequentemente, aumentará o tempo necessário para a realização da mesma.\\

As características discutidas nesta sessão servirão de base para produção do protocolo de testes e, por fim, em trabalhos futuros para o desenvolvimento da arquitetura de rede a ser implementada no projeto SpaceVANT. 