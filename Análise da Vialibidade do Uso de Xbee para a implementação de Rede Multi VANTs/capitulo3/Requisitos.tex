\mychapter{Características de um Rede Multi VANTs}
\label{Cap:Requisitos}

A utilização de redes multi VANTs se deu pela sua eficiência em tempo e custo em relação a redes de um único VANT na realização de missões dos mais variados tipos. \cite{gupta2015survey} descreve em seu trabalho as características únicas que garantem as redes multi VANTs essa vantagem competitiva.

Em \emph{Survey of Important Issues in UAV Communication Networks}, Gupta compara as características das redes de comunicação multi VANTs com redes MANET e VANET que, respectivamente, são redes para comunicação \emph{mobile} e comunicação veicular. Essa comparação é feita devido a similaridades entre as mesmas, no entanto, conclui-se que os trabalhos nessas duas áreas não atendem por completo as necessidades especificas de uma rede multi VANTs.

Por exemplo, a complexidade do modelo de mobilidade dos nós em uma rede multi VANTs não se compara ao das redes VANETs ou MANETs. Um nó em uma rede multi VANTs pode se mover de forma randômica ou ordenada, dependendo da natureza da missão a ser executada, e esse movimento pode ocorrer nas três dimensões, enquanto os nós de uma rede VANETs ou MANETs irão se mover de forma mais previsível e realizarão o movimento em apenas duas dimensões, ao longo de um rodovia como é o caso da rede VANETs. Dessa forma, as mudanças em topologia de uma rede multi VANTs serão mais frequentes.

Além da questão da mobilidade dos nós, outra característica das rede multi VANTs que influencia nas frequentes mudanças de topologia da rede seria o acesso limitado a energia e o curto tempo de voo de um VANT, algo em torno de trinta minutos com uma carga. Nós serão desligados da rede devido a baixos níveis de bateria ou mal funcionamento e outros serão adicionados a rede para substituir os nós perdidos, quando necessário.

Por outro lado, como são formadas por vários nós, as redes multi VANTs são mais confiáveis quando comparadas com redes compostas por um único VANT. Em cenário onde um dos nós da rede perde conexão, a rede pode se reorganizar e garantir a sobrevivência da mesma. No entanto, essa reorganização topológica se traduz em um consumo extra de energia na rede por completo, dessa forma, reafirmando a importância de uma rede implementada de forma otimizada quanto ao consumo energético. 

Portanto, a rede multi VANTs possuem características especificas que a diferencia de outra redes tais quais a complexidade da mobilidade dos nós, o poder de reorganização da rede para garantir a sobrevivência e escalabilidade da mesma e, por consequência disso tudo, a necessidade de otimização do consumo energético para garantir maior tempo de operação do sistema como um todo.