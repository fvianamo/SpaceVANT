%% Resumo %%

\mychapterast{Resumo}

Esse trabalho é complementar ao desenvolvimento de uma arquitetura de rede e estratégia de busca a ser implementada no projeto SpaceVANT realizado pelo Centro de Lançamento da Barreira do Inferno (CLBI) em parceria com a Universidade Federal do Rio Grande do Norte (UFRN). Aqui é discuta a possibilidade de utilizar a tecnologia XBee, mais especificamente o módulo XBee PRO S3 900HP, para a implementação da rede de comunicação de um sistema multi VANTs para varredura de uma área de impacto de foguetes. 

Primeiramente, um breve descrição do projeto SpaceVANT é realizado no intuito de familiarizar o leitor com a problemática que guia o projeto, essa descrição inclui uma comparação entre a solução que está sendo utilizada pelo CLBI atualmente e a solução proposta. Logo após são descritas as funcionalidades do módulo XBee PRO S3 900HP o qual pretende-se utilizar para implementação.

Por fim, o protocolo utilizado para realização dos testes de qualidade da rede XBee é apresentado e seus resultados discutidos na ótica de um sistema multi VANT visando a operação de varredura de área de impacto de foguetes. 
 
\textbf{Palavras-chave}: Rede de Comunicação; VANT; Rede Multi VANTs; Estratégia de Busca; Xbee.

\mychapterast{Abstract}

This work is complementary to the development of a network architecture and a search strategy to be implemented on the project SPACEVANT realized by the Barreira do Inferno Launch Center (CLBI) in partnership with Rio Grande do Norte Federal University (UFRN). The posibility of using XBee technology, more preciselly the XBee PRO S3 900HP model, to implement the comunication network on a multi UAVs system for rocket impact area surveilance is discussed here. 

First of all, a brief explanation on the SpaceVANT is done in order to familiarize the reader with the problematic that guide the project, that includes a parallel between the solution as it is nowadays at the CLBI and the proposed one. Right after, XBee PRO S3 900HP functionalities are presented, as it is the model intended to be used on the proposed solution.

Lastly, the experimental protocol used for doing the quality tests on the XBee network is presented and its results discussed on the multi UAVs systems point of view, aiming the rocket impact area search operation.   



\textbf{Keywords}: Comunication Network; UAV; Multi UAVs Network; Search Strategy; Xbee.