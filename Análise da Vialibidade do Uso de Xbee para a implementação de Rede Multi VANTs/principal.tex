%%
%% Arquivo principal para:
%% - teses de doutorado
%% - dissertaes de mestrado
%% - exames de qualificao de mestrado e doutorado
%%
%% NOTA: A PUBLICAO DESTE MODELO VISA APENAS ORIENTAR OS PS-GRADUANDOS
%% NA PREPARAO DE SEUS TEXTOS. O PPgEE DA UFRN NO PROV ASSISTNCIA NO
%% USO DAS FERRAMENTAS NECESSRIAS AO USO DESTE MODELO (LATEX, XFIG, ETC.)
%% 1
%% Adelardo Medeiros, dezembro de 2005.

% DEFINIES GLOBAIS

% Esta primeira linha d informaes gerais sobre o documento.
% PARA A VERSO FINAL:
% papel A4, letra grande (12pt), openr	sight (captulos s iniciam em
% pgina direita, se necessrio incluindo uma pgina em branco),
% twoside (o documento vai ser impresso em frente e costa)
%\documentclass[a4paper,12pt,openright,twoside]{book}
% PARA A QUALIFICAO E PARA A VERSO INICIAL:
% papel A4, letra grande (12pt), openany (captulos iniciam em
% qualquer pgina), oneside (o documento vai ser impresso s na frente)
\documentclass[a4paper,12pt,openany,oneside]{book}

% PACOTES OBRIGATRIOS
\usepackage{epigraph}
\usepackage{setspace}

\usepackage[utf8]{inputenc}
%\usepackage{lipsum}
% Use estes pacotes para poder digitar diretamente as letras com acento
% e para que a hifenizao funcione corretamente
%\usepackage[latin1]{inputenc}
\usepackage{ae}
% Para usar fontes standard ao invs das do LaTeX (gera melhores PDFs)
\usepackage{pslatex}
% Para a hifenizao em portugus
\usepackage[portuguese, brazil]{babel}
\usepackage[utf8]{inputenc}
\usepackage[T1]{fontenc}

% Para que os primeiros pargrafos das sees tambm sejam indentados
\usepackage{indentfirst}
% Para poder incluir grficos (figuras)
\usepackage{graphicx}
% Para poder fazer glossrio ou lista de smbolos
% Use a segunda opo se quiser incluir na definio do smbolo a
% pgina e/ou a equao onde ela foi definida
\usepackage[portuguese,noprefix]{nomencl}
%\usepackage[portuguese,noprefix,refeq,refpage]{nomencl}
% Para permitir espaamento simples, 1 1/2 e duplo
\usepackage{setspace}
% Para usar alguns comandos matemticos avanados muito teis
\usepackage{amsmath}
% Para poder usar o ambiente "comment"
\usepackage{verbatim}
% Para poder ter tabelas com colunas de largura auto-ajustvel
\usepackage{tabularx}
% Para executar um comando depois do fim da pgina corrente
\usepackage{afterpage}
% Para formatar URLs (endereos da Web)
\usepackage{url}
% Para reduzir os espaos entre os tens (itemize, enumerate, etc.)
% Este pacote no faz parte da distribuio padro do LaTeX.
\usepackage{noitemsep}
% Para as citaes bibliogrficas
\usepackage[abbr]{harvard}	% As chamadas so sempre abreviadas
\harvardparenthesis{square}	% Colchetes nas chamadas
%\harvardyearparenthesis{round}	% Parntesis nos anos das referncias
\renewcommand{\harvardand}{e}	% Substituir "&" por "e" nas referncias

% PACOTES OPCIONAIS

% Para poder incluir arquivos Postscript com cores (do Xfig, por exemplo)
\usepackage{color}
% Para ter clulas em tabelas que ocupam mais de uma linha
\usepackage{multirow}
% Para poder ter tabelas longas em mais de uma pgina
%\usepackage{longtable}
% Para poder escrever partes do texto em "n" colunas
%\usepackage{multicol}
% Se voc quiser personalizar os cabealhos das pginas
%\usepackage{fancyheadings}
% Para incluir algoritmos e listagens de cdigos
\usepackage{listings}
% Captulos com ttulos em um formato "decorado"
\usepackage{capitulos}

\hyphenation{skinRGBDetect}

%FIGURAS LADO A LADO
\usepackage{subfigure}

% NOVOS COMANDOS

% As definies dos novos comandos esto agrupadas no arquivo "comandos.tex"
% Esta separao  opcional: se voc preferir, pode por as definies
% diretamente neste arquivo
% newcommand define novos comandos, que podem passar a ser usados da
% mesma forma que os comandos LaTeX de base.

% Implica��o em f�rmulas
\newcommand{\implica}{\quad\Rightarrow\quad} %Meio de linha
\newcommand{\implicafim}{\quad\Rightarrow}   %Fim de linha
\newcommand{\tende}{\rightarrow}
\newcommand{\BibTeX}{\textsc{B\hspace{-0.1em}i\hspace{-0.1em}b\hspace{-0.3em}}\TeX}

% Fra��o com parentesis
\newcommand{\pfrac}[2]{\left(\frac{#1}{#2}\right)}

% Transformada de Laplace e transformada Z
%\newcommand{\lapl}{\makebox[0cm][l]{\hspace{0.1em}\raisebox{0.25ex}{-}}\mathcal{L}}
\newcommand{\lapl}{\pounds}
\newcommand{\transfz}{\mathcal{Z}}

% N�o aparecer o n�mero na primeira p�gina dos cap�tulos
\newcommand{\mychapter}[1]{\chapter{#1}\thispagestyle{empty}}

% Os cap�tulos sem n�mero
\newcommand{\mychapterast}[1]{\chapter*{#1}\thispagestyle{empty}
\chaptermark{#1}
\afterpage{\markboth{\uppercase{#1}}{\rightmark}}
\markboth{\uppercase{#1}}{}
}

% Se��es sem n�mero
\newcommand{\mysectionast}[1]{\section*{#1}
\addcontentsline{toc}{section}{#1}
\markright{\uppercase{#1}}
}

% No tabularx, as celulas devem ser centradas verticalmente
\renewcommand{\tabularxcolumn}[1]{m{#1}}

% C�lulas centralizadas horizontalmente no tabularx
\newcolumntype{C}{>{\centering\arraybackslash}X}

%% Abrevia figuras e tabelas
%\def\figurename{Fig.}
%\def\tablename{Tab.}


%
% As margens
%

% Dire\c{c}\~{a}o horizontal

% Margem interna
% Nas pginas mpares
\setlength{\oddsidemargin}{3.5cm}         % Margem real desejada
% Nas pginas pares
\setlength{\evensidemargin}{2.5cm}        % Margem real desejada
% Largura do texto
\setlength{\textwidth}{15cm}
% As margens laterais no LaTeX so sempre 1 polegada maiores do que as
% fixadas. Se foi fixada \setlength{\oddsidemargin}{3.5cm}, a margem
% real seria de 3.5+2.54=6.04cm. Para permitir que voc no tenha que
% fazer esta conta, pode usar o nmero desejado nas linhas anteriores
% e a gente subtrai 1in nas prximas linhas
\addtolength{\oddsidemargin}{-1in}
\addtolength{\evensidemargin}{-1in}
% Note que a margem direita no  fixada diretamente:
% ela  obtida subtraindo-se os outros valores da largura da pgina.
% 3.5+15+x=21cm (largura A4) -> x = margem externa = 2.5cm

% Dire\c{c}\~{a}o vertical

% Margem superior (entre o topo da folha e o cabealho), altura do
% cabealho e distncia entre o fim do cabealho e o incio do texto
\setlength{\topmargin}{2.0cm}             % Margem real desejada
\setlength{\headheight}{1.0cm}
\setlength{\headsep}{1.0cm}
% Altura do texto (sem cabealho e rodap)
\setlength{\textheight}{22.7cm}
% Distncia do fim do texto ao rodap
\setlength{\footskip}{1.0cm}
% A margem superior no LaTeX  sempre 1 polegada maior do que a
% fixada. Se foi fixada \setlength{\topmargin}{2.0cm}, a margem
%real seria de 2.0+2.54=4.54cm. Para permitir que voc no tenha que
% fazer esta conta, pode usar o nmero desejado na linha anterior
% e a gente subtrai 1in na prxima linha
\addtolength{\topmargin}{-1in}
% Note que a margem inferior no  fixada diretamente:
% ela  obtida subtraindo-se os outros valores, sem incluir o
% "footskip", da altura da pgina.
% 2.0+1.0+1.0+22.7+x=29.7cm (altura A4) -> x = margem inferior = 3cm

%
% O estilo das referncias bibliogrficas
%

\bibliographystyle{ppgee}

%
% O espaamento entre linhas
%

% As pginas iniciais so sempre em espasamento simples
\singlespacing

% Para a criao do glossrio (ou lista de smbolos)
\makeglossary

% Lista de arquivos a serem processados. Estes comandos "includeonly" so
% dispensveis e devem obrigatoriamente ser comentados na hora de gerar o
% documento definitivo. Eles servem para compilar apenas parte do documento.
%  til, durante a redao, para que no se tenha de compilar todo o
% documento a cada vez que se faz uma alterao. Por exemplo, se eu estou
% fazendo alteraes na dedicatria e as outras partes no tm interesse no
% momento, posso incluir (descomentar) a linha "\includeonly{preambulo}"
%\includeonly{rosto}
%\includeonly{catalograficos}
%\includeonly{preambulo}
%\includeonly{resumos}
%\includeonly{capitulo1/introducao}
%\includeonly{capitulo2/metodologia}
%\includeonly{capitulo3/software}
%\includeonly{capitulo4/estudo}
%\includeonly{capitulo5/conclusao}
%\includeonly{apendice/apendice}

% ? inicio codigo fonte
\usepackage{listings}
\lstset
{
numbers=left,
stepnumber=5,
firstnumber=1,
numberstyle=\tiny,
extendedchars=true,
breaklines=true,
frame=tb,
basicstyle=\footnotesize,
stringstyle=\ttfamily,
showstringspaces=false
}
\renewcommand{\lstlistingname}{Listagem}
\renewcommand{\lstlistlistingname}{Lista de C\'{o}digos}
%=========================
% ? fim codigo fonte

% Inicia o texto
\begin{document}

% Paginas iniciais (sem numerao)
\pagestyle{empty}

% P\'{a}gina de rosto (capa interna)
%
% ********** Pagina de Rosto
%

% titlepage gera paginas sem numera\c{c}ao
\begin{titlepage}

\begin{center}

\small

% O comando @{} no ambiente tabular x  para criar um novo delimitador
% entre colunas que no a barra vertical | que  normalmente utilizada.
% O delimitador desejado vai entre as chaves. No exemplo, no h nada,
% de modo que o delimitador  vazio. Este recurso est sendo usado para
% eliminar o espao que geralmente existe entre as colunas
\begin{tabularx}{\linewidth}{@{}l@{}C@{}r@{}}
% A figura foi colocada dentro de um parbox para que fique verticalmente
% centralizada em relao ao resto da linha
\parbox[c]{3cm}{\includegraphics[width=\linewidth]{LogoUFRN}} &
\begin{center}
\textsf{\textsc{Universidade Federal do Rio Grande do Norte\\
Centro de Tecnologia\\
Departamento de Engenharia de Computa\c{c}\~{a}o e Automa\c{c}\~{a}o\\
Curso de Engenharia de Computa\c{c}\~{a}o}}
\end{center}
\end{tabularx}


% O vfill  um espao vertical que assume a mxima dimenso possvel
% Os vfill's desta pgina foram utilizados para que o texto ocupe
% toda a folha
\vfill

\LARGE

\textbf{ANÁLISE DA VIABILIDADE DO USO DE XBEE PARA A IMPLEMENTAÇÃO DE UMA REDE MULTI VANTS}

\vfill

\Large

\textbf{Filipe Viana Monteiro}

\vfill
%
\normalsize

Orientador: Prof. Dr. Pablo Javier Alsina
% Se no houver co-orientador, comente a pr\'{o}xima linha
%\\[2ex] Co-orientador: Prof. Dr. Beltrano Catandura do Amaral

\vfill



%\textbf{Disserta\c{c}\~{a}o de Mestrado}
%\textbf{Tese de Doutorado}
%apresentada ao Programa de P\'{o}s-Gradu\c{c}\~{a}o em Engenharia El\'{e}trica da UFRN
%(\'{a}rea de concentra\c{c}\~{a}o: Automa\c{c}\~{a}o e Sistemas)
%(\'{a}rea de concentra\c{c}\~{a}o: Telecomunica\c{c}\~{o}es)
%como parte dos requisitos para obten\c{c}\~{a}o do t\'{\i}tulo de
%Mestre em Ci\^{e}ncias.}
%Doutor em Ci\^{e}ncias.}

\vfill

\large

Natal/RN

Dezembro de 2016

\end{center}

\end{titlepage}



% Ficha catalografica: os dados catalogrficos devem ser fornecidos
% pela BCZM.
% S\'{o} s\~{a}o inclu\'{\i}dos na vers\~{a}o final da tese ou disserta\c{c}\~{a}o. N\~{a}o s\~{a}o
% inclu\'{\i}dos nem na proposta de tema de qualificao nem na vers\~{a}o
% preliminar da tese ou disserta\c{c}\~{a}o: nestes casos, comente a pr\'{o}xima linha.

%%
% ********** Ficha Catalogr�fica
%

\newpage

\begin{center}

% Aqui n�o se usou \vfill porque o \vfill � constru�do internamente com
% o comando \vspace. Espa�os verticais no in�cio da folha com \vspace
% s�o ignorados. Para que isto n�o ocorra deve-se usar o \vspace*
% \vspace*{\fill} � como se fosse um \vfill*
\vspace*{\fill}

Divis�o de Servi�os T�cnicos\\[1ex]
Cataloga��o da publica��o na fonte.
UFRN / Biblioteca Central Zila Mamede

\vspace{2ex}

\begin{tabular}{|p{0.9\linewidth}|} \hline
\\
Pereira, Fulano dos Anz�is.\\
\hspace{1em} Sobre a Prepara��o de Propostas de Tema, Disserta��es
e Teses no Programa de P�s-Gradua��o em Engenharia El�trica da UFRN /
Fulano dos Anz�is Pereira - Natal, RN, 2006 \\
\hspace{1em} 23 p. \\
\\
\hspace{1em} Orientador: Sicrano Matosinho de Melo \\
\hspace{1em} Co-orientador: Beltrano Catandura do Amaral \\
\\
\hspace{1em} Tese (doutorado) - Universidade Federal do Rio Grande do Norte.
Centro de Tecnologia. Programa de P�s-Gradua��o em Engenharia El�trica. \\
\\
\hspace{1em} 1. Reda��o t�cnica - Tese. 2. \LaTeX - Tese.
I. Melo, Sicrano Matosinho de. II. Amaral, Beltrano Catandura do.
III. T�tulo. \\
\\
RN/UF/BCZM \hfill CDU 004.932(043.2) \\ \hline
\end{tabular} 

\end{center}


% Assinaturas da banca, dedicatria e agradecimentos
% S\'{o} s\~{a}o inclu\'{\i}dos na verso final da tese ou dissertao. N\~{a}o s\~{a}o
% inclu\'{\i}dos nem na proposta de tema de qualifica\c{c}\~{a}o nem na vers\~{a}o
% preliminar da tese ou dissertao: nestes casos, comente a pr\'{o}xima linha.
% ********** P\'{a}gina de assinaturas
%

\begin{titlepage}
%
\begin{center}
%

%
\Large \textbf{Análise da Viabilidade do Uso de Xbee para a Implementação de um Rede Multi VANTs}


\vfill

\Large \textbf{Filipe Viana Monteiro}

\bigskip
\bigskip
\bigskip
\bigskip

\normalsize

Orientador: Prof. Dr. Pablo Javier Alsina

\vfill

\hfill
\parbox{0.5\linewidth}{
% Descomente as opes que se aplicam ao seu caso
\textbf{Monografia}
apresentada \`{a} Banca Examinadora do Trabalho de Conclus\~{a}o do Curso de
Engenharia de Computa\c{c}\~{a}o, em cumprimento \`{a}s exig\^{e}ncias
legais como requisito parcial \`{a} obten\c{c}\~{a}o do t\'{\i}tulo de Engenheiro de
Computa\c{c}\~{a}o.}

\vfill

\large

Natal/RN

Dezembro de 2016

\end{center}
\end{titlepage}

%
% ********** Dedicat\'{o}ria
%

% A dedicat\'{o}ria n\~{a}o \'{e} obrigat\'{o}ria. Se voc\^{e} tem algu\'{e}m ou algo que teve
% uma import\^{a}ncia fundamental ao longo do seu curso, pode dedicar a ele(a)
% este trabalho. Geralmente n\~{a}o se faz dedicat\'{o}ria a v\'{a}rias pessoas: para
% isso existe a se\c{c}\~{a}o de agradecimentos.
% Se n\~{a}o quiser dedicat\'{o}ria, basta excluir o texto entre
% \begin{titlepage} e \end{titlepage}

\begin{titlepage}

\vspace*{\fill}

\hfill
\begin{minipage}{0.5\linewidth}
\begin{flushright}
\large\it
Aos meus pais, Américo Monteiro Filho e Rosalba Viana Monteiro, que com todos os esforços, puderam me proporcionar a melhor educação possível.

\end{flushright}
\end{minipage}

\vspace*{\fill}

\end{titlepage}

%
% ********** Agradecimentos
%

% Os agradecimentos n\~{a}o s\~{a}o obrigat\'{o}rios. Se existem pessoas que lhe
% ajudaram ao longo do seu curso, pode incluir um agradecimento.
% Se n\~{a}o quiser agradecimentos, basta excluir o texto ap\'{o}s \chapter*{...}

\chapter*{Agradecimentos}
\thispagestyle{empty}

\begin{trivlist}  \itemsep 2ex

\item Primeiramente tenho que agradecer ao meus pais, Américo Monteiro Filho e Rosalba Viana Monteiro, que sempre me possibilitaram o melhor ensino que nos era acessível. Nunca esquecerei o esforço que fizeram, e ainda fazem, para educar a mim e a minha irmã Andreza. Espero um dia poder me torna um pai tão bom quanto vocês.
  
\item Também devo agradecer aos amigos que sempre me apoiaram durante minha trajetória académica, seja ajudando a entender os conteúdos de sala de aula ou me ajudando a esquecer um pouco desses conteúdos e aliviar um pouco minha cabeça de toda a pressão que é a vida universitária.

\item Por fim tenho que agradecer, em especial, a excepcional turma de Engenharia da Computação formada em 2013, a qual faço parte. Acho que a nossa união foi um ponto que ajudou bastante na nossa formação. Nunca esquecerei das noites de estudos e projetos no DCA. E que venham muitas terças da pizza naquele mesmo lugar para comemorar nossas conquistas daqui pra frente.

\item Muito Obrigado a todos vocês.

\end{trivlist}

%\newpage

%\vspace*{\fill}
%\setlength{\epigraphrule}{1pt}
%\onehalfspacing
%\setlength{\epigraphwidth}{.95\textwidth}

%\begin{epigraphs}
%\qitem{
%    \textit{
%        \linebreak "Já dizia Tobias"
%        }
%    }
%{\textsc{Autor Desconhecido}}
%\end{epigraphs}



%\vspace*{\fill}


%
% O espa\c{c}amento entre linhas
%

% PARA A VERS\~{A}O FINAL:
% Deve ser usado espa\c{c}amento simples nas p\'{a}ginas de texto
%\singlespacing
% PARA A QUALIFICA\c{C}\~{A}O E PARA A VERS\~{A}O INICIAL:
% Deve ser usado espa\c{c}amento 1 1/2 nas pginas de texto
\onehalfspacing

% Resumo/Abstract
%% Resumo %%

\mychapterast{Resumo}

Este trabalho apresenta o desenvolvimento e explicação detalhada de uma aplicação que integra elementos de sistemas industriais (sensores, controladores e transmissores, por exemplo) com sistemas supervisórios, tornando possível a obtenção de dados desses elementos e controle das grandezas por eles medidas. O trabalho foi realizado com base em uma planta industrial que simula sistemas petrolíferos, presente no Laboratório de Avaliação de Medição em Petróleo (LAMP), que comporta elementos de instrumentação que necessitam serem lidos e manipulados através de Controladores Lógicos Programáveis (CLPs) e Sistemas de Supervisão e Aquisição de Dados (SCADA). O projeto realizado resume-se em captar os dados de sensores presentes nesse sistema industrial através de um CLP e um microcontrolador e realizar a comunicação destes com uma aplicação de supervisão desenvolvida no software Elipse SCADA. Foram utilizados CLPs da fabricante WEG modelo TPW03 e um microcontrolador da NOVUS modelo N2000. A comunicação foi realizada utilizando protocolo Modbus.



\textbf{Palavras-chave}: CLP; SCADA; Comunicação; TPW03; N2000; Modbus.

\mychapterast{Abstract}

This work presents the development and detailed explanation of an application that integrates elements of industrial systems (sensors, controllers and transmitters, for example) with supervisory systems, making it possible to obtain data of these elements and control the quantities they measures. The work was carried out on an industrial plant that simulates petroleum systems present in the Laboratório de Avaliação de Medição em Petróleo (LAMP), which includes instrumentation elements that need to be read and handled by programmable logic controllers (PLCs) and a Supervisory Control and Data Acquisition (SCADA) system. The project can be summarized by capturing data from sensors present in this industrial system through a PLC and a microcontroller and perform communication of these elements with a supervision application developed on Elipse SCADA software. We used WEG's PLC TPW03 and a NOVUS' microcontroller N2000 model. The communication was performed using Modbus protocol.


\textbf{Keywords}: PLC; SCADA; Communication; TPW03; N2000; Modbus.

% P\'{a}ginas introdut\'{o}rias (com numera\c{c}\~{a}o romana)
\frontmatter

% Lista de conte\'{u}do (gerado automaticamente)
\addcontentsline{toc}{chapter}{Sum\'{a}rio}
\tableofcontents

%%% Lista de figuras (gerada automaticamente)
\cleardoublepage
\addcontentsline{toc}{chapter}{Lista de Figuras}
\listoffigures

% Lista de tabelas (gerada automaticamente)
%\cleardoublepage
%\addcontentsline{toc}{chapter}{Lista de Tabelas}
%\listoftables

% Gloss\'{a}rio (gerado automaticamente - veja entradas em
% introducao/introducao.tex e em estilo/estilo.tex)

%\cleardoublepage
%\renewcommand{\nomname}{Lista de S\'{\i}mbolos e Abreviaturas}
%\markboth{\MakeUppercase{\nomname}}{\MakeUppercase{\nomname}}
%\addcontentsline{toc}{chapter}{\nomname}

% O argumento opcional do comando \printglossary  a largura deixada
% para os s\'{\i}mbolos no gloss\'{a}rio. Se seus s\'{\i}mbolos s\~{a}o "largos", como
% neste exemplo, \'{e} melhor por mais espa\c{c}o do que o 1cm que  reservado
% por default
%printglossary{1.5cm}% P\'{a}ginas do texto principal (com cabealho)
\mainmatter
\pagestyle{headings}

% Para facilitar a organiza\c{c}\~{a}o, foi criado um diretorio para cada
% captulo do documento, pois assim os arquivos das figuras ficam
% classificados por captulos

% Gerar Refer\^{e}ncias - bibtex principal
% Gerar Gloss\'{a}rio - makeindex -s nomencl.ist -o principal.gls principal.glo

% Comente Introduo abaixo e descomente seu capitulo

\mychapter{Introdução}
\label{Cap:introducao}

A utilização de veículos não tripulados já é bastante evidente em aplicações tanto civis quanto militares. Podem-se encontrar veículos dessa categoria substituindo a presença humana em situações onde há risco à integridade física ou quando o acesso é simplesmente impossível \cite{UAVSurveypt1}. Dentre os veículos não tripulados, temos a categoria de veículos aéreos não tripulados (VANTs) que são usados largamente para realização de filmagens aéreas a baixo custo. Com o investimento de algumas centenas de dólares, qualquer pessoa pode começar a produzir imagens aéreas utilizando VANTs comerciais. O mercado está repleto de modelos comerciais disponíveis para o público em geral, como por exemplo os quadrirrotores fabricados pela DJI\textsuperscript{\texttrademark}, o recém anunciado \emph{Karma} fabricado pela GoPro\textsuperscript{\texttrademark}, entre outros.

As aplicações para veículos aéreos não tripulados não se restringe ao uso civil ou para gravação de imagens aéreas, esta plataforma já vem sendo utilizada também em aplicações militares. Ao aliar o poder da plataforma em questão com outras tecnologias, como por exemplo o processamento digital de imagens, problemas mais complexos podem ser resolvidos. 

Um problema que pode ser solucionado com a utilização de VANTs dotados de ferramentas para processamento digital de imagem seria a identificação de embarcações não autorizadas em área de impacto de foguetes, problema esse relevante ao Centro de Lançamento da Barreira do Inferno (CLBI) localizado em Natal no Rio Grande do Norte. 

Em parceria com a Universidade Federal do Rio Grande do Norte (UFRN), através do projeto de pesquisa SPACEVANT, o CBLI vem desenvolvendo uma solução, incluindo software e hardware, para a realização da verificação da área de impacto de foguetes de forma autônoma utilizando VANTs. 

\section{Objetivos}

Como parte do desenvolvimento dessa solução, este trabalho tem por objetivo validar as especificações técnicas do transmissor XBEE PRO S3 900HP adquirido para a implementação da rede de comunicação e a viabilidade da utilização desse tipo de equipamento no contexto de uma rede multi VANT.

A fim de realizar essa validação, foram realizados testes de potência do sinal e taxa de transferência de pacotes em uma rede \emph{mesh/ad hoc}, implementada por módulos XBee PRO S3 900HP, usando quadrirrotores \emph{Phantom 3} do modelo Standard fabricados pela DJI\textsuperscript{\texttrademark} para variar a distância entre os pontos da rede e, posteriormente, verificar os efeitos do distanciamento nos parâmetros estudados.

\section{Estrutura do Trabalho}

Após este capítulo introdutório, é apresentada uma breve descrição do projeto SpaceVANT a fim de familiarizar o leitor com o contexto deste trabalho. Em seguida, no Capítulo 3, são discutidas as características das redes multi VANTs. 

A estratégia de varredura de área desenvolvida pelo mestrando Maurício Rabello para o projeto é apresentada no Capítulo 4. 

No Capítulo 5, são apresentados os módulos XBee adquirido pelo projeto para a implementação da rede, descrevendo suas especificações técnicas e os modos de operação disponíveis. 

Em seguida, o procedimento experimental desenvolvido, bem como os equipamentos utilizados para a sua realização, são apresentados no Capítulo 6.

Por fim temos a discussão dos resultados experimentais e a conclusão do trabalho nos Capítulos 7 e 8, respectivamente.  

\mychapter{Esrtutura do Sistema}
\label{Cap:estruturaSistema}


A fim de atender aos objetivos desejados de maneira coerente ao que se observa nas instalações do laboratório, tem-se a necessidade de entender as mesmas para determinar uma forma de desenvolver o projeto posteriormente. Devido a isso, pode-se dizer que as instalações do Laboratório de Injeção do LAMP serviram como base para esse trabalho, visto que a partir delas, percebeu-se a necessidade da criação de um sistema supervisório e de interligar os elementos do projeto industrial. 

A fim de entender melhor a base do trabalho, esse capítulo apresenta inicialmente uma descrição dessas instalações e posteriormente, a descrição do sistema montado, considerando especificações técnicas, características, instrumentos e protocolos utilizados.

\section{Instalações do Laboratório}

O LAMP dispõe de um espaço de instalações físicas de uma planta do Laboratório de Monitoramento de Injeção com o objetivo de realizar ensaios de medição de vazão. Um diagrama das instalações do laboratório pode ser observado na Figura \ref{fig:Planta}. A figura consiste em uma simplificação da planta, ilustrando apenas componentes essenciais para o entendimento do sistema.



\begin{figure}[h!]
  \center
  \includegraphics[scale=0.6]{Planta.png}
  \label{fig:Planta}
  \caption{Ilustração simplificada das Instalações do Laboratório de Monitoramento de Injeção.}
\end{figure}


O protótipo é composto pelos seguintes componentes:

\begin{itemize}
    \item Dois tanques de armazenamento de líquido
    \item Aquecedor: cilindro de aquecimento de água
    \item 5 Válvulas de acionamento pneumático
    \item 3 Bombas de deslocamento positivo
    \item 18 Sensores de temperatura
    \item 2 Sensores de nível
    \item 2 Sensores de vazão
    \item 4 Chaves de nível
\end{itemize}


O sistema é operado através de dois circuitos: o primeiro circuito é responsável por aquecer a água armazenada inicialmente no tanque misturador, enquanto o segundo irá transferir a água aquecida para caixotes os quais simulam a zona de injeção. Desse modo, para que o sistema funcione corretamente, a água deve estar contida no sistema de forma que as tubulações devem estar afogadas e os tanques devem conter água também.

Um sistema de aquecimento também é contido na planta. Este é responsável por fornecer energia térmica à água utilizada nos ensaios e é composto por controladores, sensores e atuadores em uma malha fechada. Sendo assim, é imprescindível que haja uma comunicação entre esses três componentes para que o sistema funcione corretamente.

\subsection{Modo de operação do circuito 1}

Para que a planta entre em operação, a válvula pneumática de saída do tanque misturador é aberta, enquanto as outras estão fechadas. Devido à ação da bomba 1, a água circula através do aquecedor, aumentando sua temperatura e retornando ao tanque misturador. Esse processo é executado em um ciclo fechado até que a temperatura desejada para a água seja atingida. 

Para que esse circuito funcione de maneira satisfatória, os dados da temperatura da água devem ser medidos e a grandeza deve ser controlada. Para isso, um controlador dedicado é responsável por controlar o acionamento do aquecedor através das informações advindas do sensor de temperatura conectado ao tanque misturador. O controlador de processos utilizado para essa finalidade é o da fabricante NOVUS modelo N2000 que será discutido mais adiante neste capítulo.

\subsection{Modo de operação do circuito 2}

No momento em que a temperatura desejada para os testes é alcançada, a bomba 2 é acionada e a água aquecida é enviada ao circuito 2, responsável pela medição dos parâmetros de temperatura e vazão, de forma a simular um sistema petrolífero real. 

Nesse circuito, a água segue para uma tubulação que simula um poço de petróleo, enterrada dentro de dois caixotes cobertos de areia. Nessa tubulação, 16 sensores te temperatura foram distribuídos com o objetivo de identificar a variação térmica ao longo da coluna de injeção, advinda da troca de calor entre o fluido injetado e o solo. A análise de vazão é realizada no ponto de medição descrito na Figura \ref{fig:Planta}. Nesse trecho é feito o controle de vazão variando-se a vazão nas derivações dos tubos de 1" e 2" que se encontram nesse ponto. Dessa forma, um perfil de temperaturas é obtido antes e outro depois do ponto de medição de vazão. Através da variação forçada da vazão do sistema, é possível comprovar e analisar o desenvolvimento teórico do projeto, pois dessa forma pode-se obter perfis de temperaturas para variadas condições de trabalho do protótipo.

Após a passagem pelo protótipo de temperatura, o líquido é armazenado temporariamente no tanque auditor. Será configurado um nível mínimo de fluido presente no tanque misturador, que quando atingido, a bomba 3 será acionada e o fluido presente no tanque auditor será bombeado para o misturador, completando o ciclo do segundo circuito.

\section{Especificações teóricas}

Pode-se afirmar que o sistema descrito no item 2.1 serve como base para o desenvolvimento desse trabalho pois para que ocorra a captação dos dados dos elementos de instrumentação descritos anteriormente, é necessária a comunicação desses ao sistema supervisório que será desenvolvido. Em intermédio a essa comunicação, é utilizado um Controlador Lógico Programável (CLP). Um CLP é um equipamento eletrônico digital composto com hardware e software utilizado em sistemas industriais, capaz de manipular elementos desses sistemas.

Esse tipo de controlador é utilizado em larga escala na indústria para o controle de diversos tipos de sistemas pois atendem a requisitos de hardware e software aptos para serem utilizados em ambientes industriais. Neste sentido, muitos CLPs possuem sistemas operacionais de tempo real, fator de extrema importância para controlar processos de alto risco, e hardware capaz de suportar possíveis variações de pressão, temperatura e umidade, além de a maioria deles possuir uma arquitetura compacta, o que facilita o deslocamento e instalação do controlador. Outra grande vantagem da utilização de CLPs na indústria é a sua capacidade e relativa facilidade de comunicação com outros elementos da malha industrial e com sistemas supervisórios.

Sistemas de Supervisão e Aquisição de Dados (SCADA) são sistemas utilizados amplamente na indústria para controle supervisório e aquisição de dados de processos industriais. Como o próprio nome indica, esses sistemas são focados no nível de supervisão de processos, sendo puramente softwares posicionados acima dos hardwares da interface industrial que podem se comunicar com CLPs ou outros hardwares \cite{daneels1999scada} para uma melhor visualização e manipulação da malha industrial a ser controlada. O sistema SCADA necessário para os projetos futuros a esse deve, em termos simplificados, comportar uma representação visual de todo o sistema físico das instalações do LAMP, captar os dados dos elementos de instrumentação e realizar possíveis controles da planta industrial de modo remoto. Durante esse trabalho, foi proposto um sistema SCADA somente para testes de comunicação e captação de parâmetros que é descrito com mais detalhes no capítulo 3 desse trabalho.

Existem diversos tipos de CLPs que variam de acordo com cada fabricante. No projeto realizado, foi utilizado um controlador lógico programável da WEG modelo TPW-03 60HT-A e um controlador universal de processos da NOVUS modelo N2000. Para desenvolvimento do sistema supervisório, utilizou-se o software Elipse SCADA. O controlador lógico da WEG foi utilizado pois ele possui uma quantidade satisfatória de entradas e saídas digitais e também dispõe de módulos de expansões, dentre eles o modelo 8AD, o qual fornece entradas analógicas, que possibilitam a leitura dos sensores que irão atuar no projeto. O Elipse SCADA, por sua vez, mostra-se um software bastante eficiente para controle supervisório, de fácil manipulação e de maior familiaridade por parte dos engenheiros do LAMP.

\section{Instrumentos Utilizados}

O trabalho aqui descrito foi realizado nos laboratórios do LAMP e os instrumentos utilizados foram:

\begin{itemize}
    \item 2 sensores de temperatura PT100
    \item 1 CLP da WEG modelo TPW-03 60HT-A
    \item 1 módulo de expansão do tipo 8AD 
    \item 1 fonte de alimentação ($24V$)
    \item 1 controlador de processos da NOVUS, modelo N2000
    \item 2 computadores para alojamento do sistema supervisório, realização de pesquisas e testes
\end{itemize}


O ambiente de trabalho é observado na Figura 2.2.%\ref{fig:Ambiente} 

\begin{figure}[h!]
  \center
  \includegraphics[scale=0.15]{Ambiente.jpg}
  \label{fig:Ambiente}
  \caption{Ambiente de Trabalho.}
\end{figure}


\section{Características e Especificações Técnicas}

Para melhor familiarização dos equipamentos utilizados, tem-se um resumo das especificações técnicas dos controladores, dos módulos de expansão, dos softwares e protocolos utilizados no trabalho.

\subsection{Sensores PT100}

Como já mencionado anteriormente, para obter os padrões de temperatura do líquido que passa pelo sistema do laboratório de injeção no LAMP, são necessários 18 sensores de temperatura. Esses sensores são do tipo PT100 (Figura 2.3), com transmissores de modelo TR321 da fabricante Salcas. Neste trabalho, utilizou-se dois desses sensores para que os testes realizados sejam condizentes com uma futura implementação do sistema final no laboratório.

\begin{figure}[h!]
  \center
  \label{fig:Pt100}
  \includegraphics[scale=0.8]{Pt100.png}
  \caption{Sensores de temperatura PT100.}
\end{figure}

O funcionamento desses sensores baseia-se na variação da resistência ôhmica em relação à temperatura. Seu elemento sensor é composto por platina com grande grau de pureza e encapsulado em bulbos de vidro ou cerâmica, o que permite uma medição padrão de termorresistência entre -200 e 650ºC \cite{salcas2015site}. Por essas características, esses sensores são um dos mais utilizados em processos industriais. Além disso, as termorresistências são totalmente customizáveis, sendo possível alterar o limite de operação do sensor para que se obtenha medidas mais precisas dependendo da aplicação. Nesse caso, como os sinais gerados pelo transmissor estão na faixa de 4 à 20 mA, o limite de medição pode ser configurado no transmissor para que o valor mínimo/máximo de corrente gerado corresponda a um valor de temperatura próximo ao mínimo/máximo em que o sistema industrial estará sujeito.

Para a aplicação em desenvolvimento nas instalações do LAMP, a água estará sujeita a temperaturas que variam entre aproximadamente 25 e 65ºC, logo, é mais eficiente definir a faixa de operação do transmissor entre esses dois valores para que seja obtida uma melhor resolução de leitura das variáveis. Essa alteração é feita através do software \textit{TxConfig II}, disponibilizado pelo fabricante.



\subsection{WEG TPW-03}

O TPW-03 é um CLP desenvolvido pela empresa WEG Automação S.A. que possui as seguintes características, descritas no manual de instalação do produto disponibilizado pelo próprio fabricante \cite{weg2010manualinstalacao}:

\begin{itemize}
    \item Alta velocidade de processamento:\\
        Instruções básicas: $0.31\mu s$ / passos (ANDB), $0.45\mu s$ / passos (LD)
    \item Grande capacidade de memória:\\
        Capacidade de memória do programa: 4k a 16k passos. O produto possui instruções de aplicação básicas e integradas, como instruções de operação, ADD/SUB/MUL/DIV…etc. instruções de trigonometria como SIN/COS/TAN…, entrada matriz, e outras instruções como saída para display de 7 segmentos e PID.
    \item Capacidade de expansão flexível:\\
        Unidades básicas: 14/20/30/40/60 pontos digitais, pode expandir no máximo até 124 pontos digitais e 8/2 (12 bits) entrada/saída analógica.
    \item 3 portas de comunicação e 3 funções de comunicação são disponíveis no modelo avançado.
    \item Conexão com Computador:\\
        Um computador pode controlar até 255 TPW-03s.
    \item Conexão de Dados: \\
        O TPW-03 Mestre pode comunicar com até 15 TPW-03 Escravos. Cada CLP tem disponível para troca de dados 64 Bits e 8 Words.
    \item E/S Remota:\\
        O TPW-03 Mestre pode controlar as E/Ss de até 4 outros TPW-03 Escravos.
    \item Compatível com Modbus:\\
        O Protocolo Modbus está desenvolvido no TPW-03. Ideal para comunicação com Inversores e IHMs.
    \item RTC, PWM, dois VR’s (potenciômetros), memória flash e capacidade de expansão de pontos digitais e analógicos.
    \item Saída de pulso de alta velocidade de 100KHz que pode controlar um servo controlador.
    \item Contador de alta velocidade:\\
        O contador pode trabalhar um ou dois canais e como entrada de interrupção, sendo que no modo contagem com um canal, sua freqüência máxima é de 100KHz.
    \item Fácil manutenção e instalação uma vez que os blocos de terminais são plugáveis.
    \item O TPW-03 pode ser programado nas linguagens Ladder e Lista de Instruções.
    \item O Firmware pode ser atualizado diretamente via PC.
\end{itemize}

O LAMP dispõe de CLPs de modelo de modulo básico TPW-03 60HT-A, o qual possui uma alimentação AC 100 - 240V, 36 entradas digitais e 24 saídas digitais do tipo transistor. Além disso, foram utilizados módulos de expansão modelo 8AD (expansão analógica), que dispõem de oito entradas analógicas, o que permite a captação de dados de sensores.

O TPW-03 possui um software de programação em LADDER próprio, o TPW03-PCLINK. Esse software foi utilizado para configurar os parâmetros de comunicação do CLP com outros controladores e com o Elipse SCADA. Essa comunicação será comentada mais adiante neste trabalho.

\subsection{NOVUS N2000}

O controlador universal da NOVUS modelo N2000 possui os seguintes detalhes técnicos \cite{novus2014folheto}:
\begin{itemize}

    \item Aceita termopares: J, K, T, N, R, S, E, B; PT100, 4-20 mA, 0-50 mV, 0-5 Vcc, 0-10 V
    \item Saídas: relé 3 A / 250 Vca, linear 4-20 mA e pulso lógico para relés de estado sólido 
    \item Alarmes: 4 relés na versão básica 
    \item Até 2 alarmes temporizados de 0 a 6500s 
    \item Resolução na medida: 12000 níveis 
    \item Indicação de decimais nas medições de temperatura
    \item Proteção da configuração por senha de acesso
    \item Interface USB 2.0, classe CDC, protocolo Modbus RTU
    \item Fonte 24 Vcc para excitar transmissores
    \item Amostragem: 5 medidas por segundo.
    \item Alimentação: 
    - 100 a 240 Vca/cc 
    - 12 a 24 Vcc / 24 Vca 
    \item Retransmissão da PV ou SP em 0 a 20 mA ou 4 a 20 mA
    \item Função Automático/Manual, transferência \textit{bumpless}
    \item Entrada de SetPoint remoto
    \item Função LBD (Loop Break Detection)
    \item Entrada Digital
    \item Função saída segura
    \item Soft-start programável (0 a 9999 seg.)
    \item Rampas e Patamares: 7 programas de 7 segmentos cada, podendo ser concatenados até formar um programa de 49 segments. Todos os segmentos podem ser associados eventos.
    \item Auto-sintonia dos parâmetros PID
    \item Teclas em silicone 
    \item Painel frontal: IP65, Policarbonato UL94 V-2
    \item Formato 48 x 96 x 92 mm
    \item Certificações CE e UL
    
\end{itemize}

Tem-se uma amostra desse controlador disponível no laboratório e o mesmo foi utilizado a fim de se comunicar com o sistema supervisório e com o TPW-03 através de comunicação serial RS485 com protocolo Modbus RTU. Esse controlador possui a vantagem de ser leve e de pequenas dimensões, além de dispor de uma fácil programação de seus parâmetros, sendo estes configuráveis através de suas teclas, no próprio aparelho, não necessitando de um ambiente de programação. Além disso, esse controlador possui entradas analógicas, que permitem também a leitura de sensores do tipo PT100, que são utilizados no trabalho.


\subsection{Elipse SCADA}

O Elipse SCADA é um software de monitoramento e programação de processos muito utilizado em linhas de produção e ambientes industriais. Nele os dados dos sistemas podem ser apresentados em tempo real de forma gráfica \cite{elipse2015manual}, o que permite uma análise eficiente e respostas rápidas. O programa conta com uma ferramenta denominada \textit{Organizer} que permite uma organização das variáveis e propriedades do sistema de forma amigável, auxiliando na organização e manipulação dos parâmetros do sistema através de \textit{displays}, botões, gráficos, etc. Outro fator primordial para a utilização desse software no projeto aqui descrito, é a possibilidade de integrar o mesmo a controladores e unidades remotas de diversas maneiras, dentre elas, via protocolo Modbus, com comunicação RS485 ou Ethernet.

\subsection{Protocolo Modbus}

O Modbus é um tipo de protocolo de comunicação, proposto pela Modicon Corporation. Esse protocolo define uma estrutura de mensagem que diferentes controladores podem reconhecer e utilizar independente do tipo de rede em que eles estão conectados. Ele descreve o processo que um controlador utiliza para requisitar acesso a outro dispositivo, como ele vai responder à pedidos de outros dispositivos e como erros serão detectados e reportados. Além disso, o protocolo estabelece um formato comum de \textit{layout} e conteúdos de campos mensagem \cite{modbus1996manual}.

Em uma rede Modbus, o protocolo determina como cada controlador saberá o endereço de cada dispositivo, reconhecer a mensagem adereçada à ele, determinar o tipo de ação que será tomada e extrair o dado ou outra informação contida na mensagem. Se uma resposta é requisitada, o controlador vai construir a mensagem de resposta e mandar utilizando o próprio protocolo Modbus\cite{modbus1996manual}.

Os controladores da rede Modbus utilizam uma comunicação mestre-escravo, na qual apenas um dispositivo (o mestre) pode iniciar a transmissão dos dados por meio de uma requisição aos outros dispositivos (escravos). Os escravos, respondem enviando o dado requisitado pelo mestre ou realizando a ação que o dispositivo mestre desejou. No trabalho aqui realizado, o dispositivo mestre da rede de comunicação montada é o PC, o qual contém o processador \textit{host} da aplicação feita no Elipse SCADA. Os escravos correspondem ao CLP TPW-03 e ao controlador de processos N2000.

O mestre pode tanto adereçar mensagens a escravos individuais ou em \textit{broadcast}, para todos os escravos. Os outros dispositivos retornam a mensagem de resposta às requisições do mestre de forma individual. 

A forma com que o protocolo estabelece o formato de requisição do mestre se dá por indicar ao mestre o endereço do dispositivo ao qual ele deseja se comunicar, o código da função que define a ação a ser executada, algum possível dado a ser enviado e um campo de checagem de erro. A mensagem de resposta do escravo também é construída utilizando o mesmo protocolo. Ela contém campos confirmando a ação tomada, qualquer dado a ser retornado e um campo de checagem de erro. Se algum erro tiver ocorrido no recebimento da mensagem, ou se o escravo for incapaz de realizar a ação requisitada, o mesmo construirá uma mensagem de erro e enviará como resposta. A Figura 2.4 demonstra um esquema da comunicação entre um dispositivo mestre e um escravo.

\begin{figure}[h!]
  \center
  \includegraphics[scale=1.0]{QueryResponse.png}
  \label{fig:QueryResponse}
  \caption{Requisição-resposta entre mestre e escravo.}\cite{modbus1996manual}
\end{figure}
\subsubsection{Modos de Operação}

Uma rede Modbus padrão pode operar segundo dois modos de transmissão: ASCII ou RTU. Esses modos definem como a informação será contida e decodificada nos campos de mensagem. Quando os controladores são dispostos a se comunicar utilizando o modo ASCII \textit{(American Standard Code for Information Interchange)}, cada byte em uma mensagem é enviado como dois caracteres ASCII. A maior vantagem desse modo é que ele permite que intervalos de cerca de um segundo entre caracteres ocorram sem causar um erro. A Tabela \ref{tab:ASCII} demonstra o formato de 1 byte em operação no modo ASCII.

\begin{table}[]
\centering
\resizebox{\textwidth}{!}{%
\begin{tabular}{|c|c|}
\hline
 & \cellcolor[HTML]{C0C0C0}Modo ASCI \\ \hline
\cellcolor[HTML]{C0C0C0}Codificação & \begin{tabular}[c]{@{}c@{}}Hexadecimal, caracteres ASCII 0–9, A–F\\ Um caracter hexadecimal contido em cada caracter ASCII da mensagem\end{tabular} \\ \hline
\cellcolor[HTML]{C0C0C0}Bits por Byte & \begin{tabular}[c]{@{}c@{}}1 bit de início \\ 7 bits de dados\\ 1 bit de paridade; nenhum bit caso sem paridade\\ 1 bit de parada, se paridade é usada; 2 bits caso sem paridade\end{tabular} \\ \hline
\cellcolor[HTML]{C0C0C0}Checagem de Erro & Longitudinal Redundancy Check (LRC) \\ \hline
\end{tabular}%
}
\caption{Formato de Byte no Modo ASCII}
\label{tab:ASCII}
\end{table}


No modo de operação RTU \textit{(Remote Terminal Unit)}, cada byte (8 bits) presente em uma mensagem é composto por dois conjuntos de 4 bits de caracteres hexadecimais. A maior vantagem desse modo é que por haver uma maior densidade de caracteres em uma mensagem, há uma maior taxa de transferência neste com relação ao modo ASCII (considerando uma mesma taxa de transmissão). Devido a essas vantagens, 
neste trabalho, optou-se por utilizar o protocolo Modbus em modo RTU. Neste modo, cada mensagem deve ser transmitida em um fluxo contínuo de caracteres. A Tabela \ref{tab:RTU} demonstra o formato de 1 byte no modo de operação RTU.

\begin{table}[]
\centering
\resizebox{\textwidth}{!}{%
\begin{tabular}{|c|c|}
\hline
 & \cellcolor[HTML]{C0C0C0}Modo RTU \\ \hline
\cellcolor[HTML]{C0C0C0}Codificação & \begin{tabular}[c]{@{}c@{}}8–bit binário, hexadecimal 0–9, A–F\\  Dois caracteres hexadecimais contidos em cada campo de 8–bits da mesnesagem\end{tabular} \\ \hline
\cellcolor[HTML]{C0C0C0}Bits por Byte & \begin{tabular}[c]{@{}c@{}}1 bit de início\\  8 bits de dados\\  1 bit de paridade; nenhum bit caso sem paridade\\  1 bit de parada se paridade é usada; 2 bits caso sem paridade\end{tabular} \\ \hline
\cellcolor[HTML]{C0C0C0}Checagem de Erro & Cyclical Redundancy Check (CRC) \\ \hline
\end{tabular}%
}
\caption{Formato de Byte no Modo RTU}
\label{tab:RTU}
\end{table}

Na prática, se em um sistema Modbus, determinado equipamento escravo falha, ou é separado da rede, o mestre pode identificar o equipamento falho e após o reparo, a rede pode ser conectada novamente automaticamente. Assim, pode-se dizer que o protocolo Modbus é confiável em relação à falhas desse tipo \cite{peng2008modbus}.

Além desses dois modos de operação em serial, existe também o protocolo Modbus/TCP, em que o controle de acesso ao meio se dá via CSMA-CD (em rede Ethernet) com o modelo cliente-servidor. Esse modo de operação não foi considerado nesse trabalho, visto que a comunicação serial mostrou-se suficiente para a aplicação desenvolvida.

%ZZZZZZZZZZZZZZZZZZZZZZZZZZZZZZZZZZZZZZZZZZZZZZZZZZZZZZZZZZZZZZZZZZZZZZZZZZZZZZZZZZ



\mychapter{Implementações}
\label{Cap:implementacoes}

A aplicação desenvolvida envolve a comunicação entre o CLP WEG TPW-03, o microcontrolador NOVUS N2000 e o \textit{software} de teste desenvolvido no Elipse SCADA. Os sensores serão fisicamente ligados ao CLP e ao controlador da NOVUS. Para que os valores desses elementos sejam observados pelo sistema supervisório, há a necessidade de se estabelecer a comunicação entre todos os elementos.

Como o trabalho é baseado nas instalações físicas nos laboratórios do LAMP, é necessário que um total de 17 sensores de temperatura e dois de pressão sejam lidos pelo sistema final descrito no item 2.1 desse trabalho. A fim de tornar os testes mais rápidos e simples de serem executados, foram utilizados apenas dois sensores de temperatura (PT100). Visto que a lógica de obtenção de dados dos sensores é similar tanto para somente um, quanto para um valor $N>1$ de sensores, o fato de se utilizar apenas dois sensores não compromete o objetivo final de comunicar todos os 19 sensores do sistema de simulação de poços petrolíferos do LAMP.

\section{Comunicação}
Para haver uma leitura dos sinais dos sensores através do TPW-03, há a necessidade da conexão de módulos analógicos, uma vez que o módulo padrão do TPW-03 60HT-A não dispõe dessas entradas. Devido a isso, optou-se por utilizar o módulo de expansão do tipo 8AD. Esse módulo comporta oito entradas analógicas, sendo possível a leitura de até oito sensores cada. Para um projeto que precise utilizar mais de oito sensores, até 60 canais de entradas e 10 canais de saídas analógicas podem ser expansíveis ao módulo básico do TPW-03 (para o modelo 60H) utilizando mais módulos AD.

A Figura \ref{fig:ArquiteturaComunicacao} demonstra a arquitetura da rede de comunicação do sistema considerado nesse trabalho.

\begin{figure}[h!]
  \center
  \includegraphics[scale=0.9]{ArquiteturaComunicacao.jpg}
  \label{fig:ArquiteturaComunicacao}
  \caption{Arquitetura da Aplicação.}
\end{figure}


A interligação dos sensores ao TPW-03 é feita segundo o manual de instalação do CLP \cite{weg2010manualinstalacao}. A Figura 3.2 apresenta o sistema de ligação de dispositivos externos, ao controlador. De acordo com o esquema de ligação apresentado, deve-se conectar o sensor à fonte de alimentação chaveada (utilizou-se uma fonte do tipo PSS24-W/2.5, 24Vdc 60W da WEG). Esse esquema de ligação é disponibilizado pelo próprio fabricante.

Como era necessária a conexão de apenas um sensor de temperatura (PT100) utilizou-se apenas o primeiro canal AD. Assim, em resumida explicação, conecta-se o polo positivo do PT100 à saída de tensão positiva da fonte de alimentação, o polo negativo do sensor ao pino A0 da expansão analógica do CLP, e a saída negativa da fonte de alimentação conecta-se ao pino C0 da expansão do controlador.

\begin{figure}[h!]
\centering
\includegraphics[scale=0.9]{Ligacao.jpg}
\label{fig:LigacaoAnalogico}
\caption{Especificação de ligação de dispositivos externos ao módulo de expansão TPW-03 8AD.}
\end{figure}


Para que ocorra a ligação do sensor PT100 com o microcontrolador da NOVUS, para o caso de sinais de corrente de 4 - 20mA, o fabricante indica a conexão apresentada na Figura \ref{fig:LigacaoNovus}. O pino positivo do PT100 é ligado ao pino 17 do N2000; o negativo do PT100 liga-se ao pino 22 do microcontrolador; e liga-se o pino 18 do aparelho da NOVUS ao seu pino 24.

\begin{figure}[h!]
\centering
\includegraphics[scale=1.0]{LigacaoNovus.jpg}
\caption{Especificação de ligação de dispositivos externos ao N2000 do tipo 4 - 20mA.} \cite{novus2014folheto}
\label{fig:LigacaoNovus}
\end{figure}

\subsection{Protocolo de Comunicação}

Uma vez interligados os sensores aos controladores, há a necessidade da comunicação entre ambos, juntamente com o PC. Essa comunicação é feita de maneira serial via RS-485 utilizando o protocolo Modbus modo RTU, discutido no capítulo 2. A razão da escolha desse protocolo para interligação dos elementos do sistema se deu pela sua popularidade, eficiência, confiabilidade e facilidade de implementação no Elipse SCADA.

A implementação do protocolo Modbus no software da Elipse se dá através de um \textit{driver} Modbus da própria \textit{Elipse software}, disponível para \textit{download} no próprio site da empresa. O software que comporta esse \textit{driver} funciona sempre como mestre de uma rede Modbus, o que define o PC como o mestre da comunicação e \textit{host} da aplicação SCADA.

O \textit{driver} da Elipse foi desenvolvido utilizando uma biblioteca chamada IOKit, responsável por implementar a camada física, definindo parâmetros como: porta de comunicação, \textit{baud rate}, bits de dado, paridade e bits de parada (para uma comunicação serial) \cite{iokit2009manual}. 

\section{Configurações dos parâmetros do WEG TPW-03}

O TPW-03 modelo 60HT-A possui três portas de comunicação: porta de comunicação do PC; cartão de expansão TPW-03 232RS e TPW-03 485RS; e porta de comunicação RS485. Das três, utilizou-se a primeira e a última durante os experimentos descritos nesse trabalho. Para estabelecer a comunicação via porta do PC, utiliza-se um cabo serial conectado via USB ao computador.

Essa comunicação permite realizar Download/Upload de programas em LADDER e conexão do CLP como escravo Modbus. A comunicação do CLP com o PC por essa porta deve ser estabelecida através do software da WEG: TPW03-PCLINK. Nesse software, é possível selecionar a ferramenta de conexão e se estabelecer o link entre o dispositivo e o PC conforme a Figura \ref{fig:link}.

\begin{figure}[h!]
\centering
\includegraphics[scale=0.7]{link.png}
\caption{Link do CLP com o PC via TPW03-PCLINK}
\label{fig:link}
\end{figure}

Nessa conexão define-se um \textit{baud rate} de 19200, visto que é o \textit{baud rate} padrão para as três portas \cite{weg2010manualinstalacao}[p.29]. Definiu-se dados no formato de 8 bits, sem paridade e 1 bit de parada. Esses parâmetros poderiam ser alterados contanto que os mesmos parâmetros sejam definidos em todos os dispositivos que se conectam na rede Modbus. É necessário também definir a porta de comunicação em que o CLP está conectado. No exemplo em questão, a conexão era estabelecida na COM4.

Após feito o link, é necessário programar o formato de comunicação e o \textit{baud rate} no registro especial da comunicação no CLP. Para isso, o endereço D8321 deve ser configurado com um valor correspondente à comunicação desejada. Esse registrador é definido especialmente para indicar parâmetros como o comprimento de dados, bit de paridade, \textit{stop bit} e \textit{baud rate} de acordo com a Figura \ref{fig:portaPC}.

\begin{figure}[h!]
\centering
\includegraphics[scale=0.7]{portaPC.png}
\caption{Programação de comunicação para a porta do PC ($D8321$)}\cite{weg2010manualinstalacao}[p.30].
\label{fig:portaPC}
\end{figure}

Considerando os parâmetros desejados, deve-se passar o valor binário de $10000001$ para o registrador $D8321$, isso corresponde ao valor 81 em hexadecimal. Esse valor hexadecimal é movido para esse registrador na programação em LADDER (Figura \ref{fig:LadderComunicacao}). 

\begin{figure}[h!]
\centering
\includegraphics[scale=0.65]{ladderComunicacao.png}
\caption{Programação em LADDER para configuração de comunicação.}
\label{fig:LadderComunicacao}
\end{figure}

No mesmo programa é possível definir os parâmetros para comunicação via RS485. Para isso, deve-se configurar o registrador $D8120$ com os parâmetros desejados conforme a Figura \ref{fig:rs485}. Como deseja-se os mesmos parâmetros de comunicação, a configuração permanece a mesma, dessa forma, o valor 81 hexadecimal também deve ser passado para o registrador $D8120$.

\begin{figure}[h!]
\centering
\includegraphics[scale=0.7]{rs485.png}
\caption{Programação do formato de comunicação para RS485 (D8120)}\cite{weg2010manualinstalacao}
\label{fig:rs485}
\end{figure}

Feita a programação em LADDER, realiza-se o download do programa para o CLP da WEG e o link com o software TPW03-PCLINK pode ser desfeito através da mesma ferramenta descrita na Figura \ref{fig:link}.

\subsection{Configuração do módulo 8AD}

Para que o módulo de expansão analógica 8AD funcione corretamente de acordo com a aplicação desejada, é necessário programar a memória do sistema para que o mesmo atue corretamente sobre as unidades conectadas à ele. Essa programação é feita de acordo com a Figura \ref{fig:memoria8AD} com informações do fabricante.

\begin{figure}[h!]
\centering
\includegraphics[scale=0.7]{memoria8AD.png}
\caption{Configuração do Modo de operação do módulo 8AD.}\cite{weg2010manualinstalacao}[p.64]
\label{fig:memoria8AD}
\end{figure}

Através do TPW03-PCLINK, configura-se os registradores ($D8256$ à $D8276$) com os valores desejados de operação do módulo. Na aplicação desenvolvida nesse trabalho, deseja-se apenas um módulo e que este opere em modo de entrada de corrente 4 - 20mA.    Para isso, deve-se passar o valor decimal $K = 1$ para o registrador $D8257$ que indica o número de expansões analógicas do TPW03. Além disso, o valor hexadecimal 3 deve ser movido para o registrador correspondente aos canais analógicos que serão utilizados.

Conforme a figura \ref{fig:Config8AD} indica, foi passado o valor '$H3333$' para o registrador $D8261$ para que as 4 primeiras entradas da expansão fossem habilitadas para leitura dos sensores no modo de operação 4-20mA.

\begin{figure}[h!]
\centering
\includegraphics[scale=0.68]{Config8AD.png}
\caption{Programação em LADDER para configuração do módulo 8AD.}
\label{fig:Config8AD}
\end{figure}

Observa-se também que o modo de operação 4 também indica a leitura específica de um aparelho PT100. Teoricamente, essa forma de operação também deve servir para os mesmos propósitos desejados aqui, porém os testes com essa forma de operação não foram realizados. A escolha do modo de operação 4-20mA em detrimento do modo PT100 foi feita para que caso se deseje conectar outro tipo de sensor que atue de 4-20mA ao CLP, este já esteja devidamente configurado para receber o aparelho.




\section{Configuração dos parâmetros do NOVUS N2000}

O controlador da NOVUS é capaz de se comunicar com o PC de maneira serial via RS485, também utilizando protocolo Modbus RTU. Para configurar a comunicação serial nesse controlador, deve-se alterar os valores das variáveis $bAud$ e $Addr$. Como deseja-se que todos os integrantes da comunicação tenham o mesmo \textit{baud rate} e configuração de bits, configura-se o $bAud$ com o valor 4, o que corresponde a um \textit{baud rate} de 19200, conforme indicado na tabela de registradores para comunicação serial do produto \cite{novus2014modbus}. O parâmetro $Addr$ corresponde ao identificador do controlador na rede Modbus. Nesse caso, conforme discute-se posteriormente, considera-se o N2000 como o segundo escravo da rede Modbus, atribuindo-se $Addr = 2$. A manipulação desses parâmetros do N2000 é realizada no próprio controlador, sem a necessidade de um software externo.

\section{Comunicação com o Sistema Supervisório}

Como mencionado na no item 3.1.1, a comunicação dos dois controladores com o PC se dá através do \textit{driver} Modicon Modbus em sua versão v3.19 (IOKitLib v2.054), permitindo com que a aplicação desenvolvida no Elipse se comunique com qualquer dispositivo escravo que implemente o protocolo Modbus.

Os parâmetros de comunicação Modbus podem ser configurados através do \textit{Organizer} dentro do Elipse SCADA. Seleciona-se o \textit{Driver Modbus} e na aba de opções \textit{Extras}, define-se o modo de aplicação (\textit{RTU Mode}) e com \textit{Data Address Model Offset = 0} visto que o endereçamento das memórias dos controladores utilizados iniciam-se de 0. Na aba \textit{Operations}, define-se as operações de leitura e escrita que podem ser utilizadas. Na aba \textit{Setup}, define-se a camada física (Serial para essa comunicação) e dependendo da opção escolhida nessa aba, configura-se a comunicação nas abas subsequentes. Para o caso de uma comunicação serial, define-se a porta de comunicação, o \textit{Baud Rate}, \textit{Data Bits}, \textit{Paridade} e \textit{Stop Bits}. Esses parâmetros são configurados com os mesmos valores discutidos anteriormente para os dois controladores.

Uma vez configurados os parâmetros de comunicação do TPW-03 e do N2000, pode-se fazer a ligação desses dispositivos com o Elipse SCADA e ter-se um sistema supervisório capaz de atuar sobre os controladores.




%aaaaaaaaaaaaaaaaaaaaaaaaaaaaaaaaaaaaaaaaaaaaaaaaaaaaaaaaaaaaaaaaaaaaaaaaaaaaaaaaaaaaaaaaaaaaaaaaaaaaaaaaaaaaaaaaaaaaaaaaaaaaaaaaaaaaaaaaaaaaaaaaaaaa












%ZZZZZZZZZZZZZZZZZZZZZZZZZZZZZZZZZZZZZZZZZZZZZZZZZZZZZZZZZZZZZZZZZZZZZZZZZZZZZZZZZZZZZZZZZZZZZZZZ



















\mychapter{Aplicação e Testes}
\label{Cap:aplicacaoTestes}



\mychapter{Conclus\~{a}o}
\label{Cap:conclusao}

Apesar do local de realização dos testes não ter sido o ideal e alguns dos dados coletados não corresponderem com o que era esperado, os testes descritos nesse trabalho são de relevância para o estudo que está sendo realizado pelo mestrando Maurício Rabelo e, consequentemente, para o projeto SpaceVANT.

A partir dos testes realizados na própria UFRN, mais especificamente no campo de futebol central, foi possível identificar diversos parâmetros que interferem na taxa de sucesso de envio dos pacotes na rede XBee, parâmetros esses como: tamanho do \emph{payload} do pacote sendo enviado e a configuração de \emph{baudrate} do módulo XBee.

Sendo assim, por mais que os valores especificados no manual do dispositivo não tenham sido confirmados nos testes, principalmente nos testes em solo, os resultados se mostraram promissores. No entanto, ainda é prematuro dizer que o módulo XBee PRO S3 900HP é adequado para implementação do sistema proposto pelo projeto SpaceVANT, mas acredita-se que essa premissa se confirmará com a realização de testes mais direcionados, de preferência realizados nas instalações do CLBI, analisando os efeitos de uma configuração mais refinada do módulo XBee e mensurando o impacto do tamanho da mensagem transmitida na perda de pacotes.     

 


% Refer\^{e}ncias bibliogr\'{a}ficas (geradas automaticamente)

\addcontentsline{toc}{chapter}{Refer\^{e}ncias bibliogr\'{a}ficas}
\nocite{*}
\bibliography{bibliografia}


\appendix

% Ap\^{e}ndice A
\include{apendice/apendice}

% Ap\^{e}ndice B
%\include{apendice/apendiceB}

% Ap\^{e}ndice C
%\include{apendice/apendiceC}

\end{document} 